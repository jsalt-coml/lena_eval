\documentclass[english,floatsintext,man]{apa6}

\usepackage{amssymb,amsmath}
\usepackage{ifxetex,ifluatex}
\usepackage{fixltx2e} % provides \textsubscript
\ifnum 0\ifxetex 1\fi\ifluatex 1\fi=0 % if pdftex
  \usepackage[T1]{fontenc}
  \usepackage[utf8]{inputenc}
\else % if luatex or xelatex
  \ifxetex
    \usepackage{mathspec}
    \usepackage{xltxtra,xunicode}
  \else
    \usepackage{fontspec}
  \fi
  \defaultfontfeatures{Mapping=tex-text,Scale=MatchLowercase}
  \newcommand{\euro}{€}
\fi
% use upquote if available, for straight quotes in verbatim environments
\IfFileExists{upquote.sty}{\usepackage{upquote}}{}
% use microtype if available
\IfFileExists{microtype.sty}{\usepackage{microtype}}{}

% Table formatting
\usepackage{longtable, booktabs}
\usepackage{lscape}
% \usepackage[counterclockwise]{rotating}   % Landscape page setup for large tables
\usepackage{multirow}		% Table styling
\usepackage{tabularx}		% Control Column width
\usepackage[flushleft]{threeparttable}	% Allows for three part tables with a specified notes section
\usepackage{threeparttablex}            % Lets threeparttable work with longtable

% Create new environments so endfloat can handle them
% \newenvironment{ltable}
%   {\begin{landscape}\begin{center}\begin{threeparttable}}
%   {\end{threeparttable}\end{center}\end{landscape}}

\newenvironment{lltable}
  {\begin{landscape}\begin{center}\begin{ThreePartTable}}
  {\end{ThreePartTable}\end{center}\end{landscape}}




% The following enables adjusting longtable caption width to table width
% Solution found at http://golatex.de/longtable-mit-caption-so-breit-wie-die-tabelle-t15767.html
\makeatletter
\newcommand\LastLTentrywidth{1em}
\newlength\longtablewidth
\setlength{\longtablewidth}{1in}
\newcommand\getlongtablewidth{%
 \begingroup
  \ifcsname LT@\roman{LT@tables}\endcsname
  \global\longtablewidth=0pt
  \renewcommand\LT@entry[2]{\global\advance\longtablewidth by ##2\relax\gdef\LastLTentrywidth{##2}}%
  \@nameuse{LT@\roman{LT@tables}}%
  \fi
\endgroup}


\ifxetex
  \usepackage[setpagesize=false, % page size defined by xetex
              unicode=false, % unicode breaks when used with xetex
              xetex]{hyperref}
\else
  \usepackage[unicode=true]{hyperref}
\fi
\hypersetup{breaklinks=true,
            pdfauthor={},
            pdftitle={A thorough evaluation of the Language Environment Analysis (LENATM) system},
            colorlinks=true,
            citecolor=blue,
            urlcolor=blue,
            linkcolor=black,
            pdfborder={0 0 0}}
\urlstyle{same}  % don't use monospace font for urls

\setlength{\parindent}{0pt}
%\setlength{\parskip}{0pt plus 0pt minus 0pt}

\setlength{\emergencystretch}{3em}  % prevent overfull lines

\ifxetex
  \usepackage{polyglossia}
  \setmainlanguage{}
\else
  \usepackage[english]{babel}
\fi

% Manuscript styling
\captionsetup{font=singlespacing,justification=justified}
\usepackage{csquotes}
\usepackage{upgreek}



\usepackage{tikz} % Variable definition to generate author note

% fix for \tightlist problem in pandoc 1.14
\providecommand{\tightlist}{%
  \setlength{\itemsep}{0pt}\setlength{\parskip}{0pt}}

% Essential manuscript parts
  \title{A thorough evaluation of the Language Environment Analysis (LENATM)
system}

  \shorttitle{IN PREP - LENA EVAL}


  \author{many\textsuperscript{1}}

  % \def\affdep{{""}}%
  % \def\affcity{{""}}%

  \affiliation{
    \vspace{0.5cm}
          \textsuperscript{1}   }

  \authornote{
    Correspondence concerning this article should be addressed to many, .
    E-mail:
  }


  \abstract{waiting}
  




\begin{document}

\maketitle

\setcounter{secnumdepth}{0}



\subsection{Brief introduction to LENA(R)
products}\label{brief-introduction-to-lenar-products}

\subsection{Previous validation work}\label{previous-validation-work}

\subsection{Present work}\label{present-work}

\section{Methods}\label{methods}

\subsection{Corpora}\label{corpora}

\subsection{Processing}\label{processing}

\subsection{LENA classification
accuracy}\label{lena-classification-accuracy}

\subsubsection{Speech and talker segmentation
metrics}\label{speech-and-talker-segmentation-metrics}

\subsubsection{Precision and recall}\label{precision-and-recall}

\subsection{CVC and CTC evaluation}\label{cvc-and-ctc-evaluation}

\subsection{AWC evaluation}\label{awc-evaluation}

\section{Results}\label{results}

Before starting, we provide some general observations based on the human
annotation. Silence is extremely common, constituting 79\% of the
frames. In fact, 34\% of clips contained no speech by any of the human
speaker types (according to the human annotators). As for speakers,
female adults make up 11\% of the frames, the child contributes to 4\%
of the frames, whereas male adult voices, other child voices, and
electronic voices are found in only 1\% of the frames each. Overlap
makes up the remaining 3\% of the frames. The following consequences
ensue: if frame-based accuracy is sought, a system that classifies every
frame as silence would be 79\% correct. This is of course not what we
want, but it indicates that systems adapted to this kind of speech
should tend to have low \enquote{false alarm} rates, i.e.~a preference
for being very conservative as to when there is speech. If the system
does say there is speech, then it had better say that this speech comes
from female adults, who provide a great majority of the speech. In
second place, it should be key child. Given that male adults and other
children are rare, a system that makes a lot of mistakes in these
categories may still have a good global performance, because these
categories are extremely rare.

\subsection{LENA classification accuracy: False alarms, misses,
confusion}\label{lena-classification-accuracy-false-alarms-misses-confusion}

Our first analysis is based on standard speech technology metrics, which
put errors in the perspective of how much speech there is. That is, if
10 frames are wrong in a file where there are 100 frames with speech,
this is a much smaller problem than if 10 frames are wrong in a file
where there is 1 frame with speech. In other words, these metrics should
be considered relative error metrics. One problem, however, emerges when
there is no speech whatsoever in a given file. In the speech technology
literature, this is never discussed, because most researchers working on
this are basing their analyses on files that have been selected to
contain speech (e.g., recorded in a meeting, or during a phone
conversation). We still wanted to take into account clips with no speech
inside because it is key for our research goals: We need systems that
can deal well with long stretches of silence, because we want to measure
how much speech children hear. Indeed, as mentioned above, 38\% of our
clips had no speech whatsoever. In these cases, the false alarm and
confusion metrics are undefined. It also occurred that there was just a
little speech; in this case, the denominator is very small, and
therefore the ratio for these two metrics ended up being a very large
number. Since the presence of outliers violate a basic assumption of
regression models, and outliers greatly impact means, we declared as NA
any metric that was 2 SD above the mean over all clips. Please note that
this leads to an overestimation of LENA's performance, because clips
where the relative error rate is very high are removed from
consideration. Also, preliminary analyses revealed that performance was
lower when near and far were collapsed together (i.e., CHN and CHF were
mapped onto a single CH category), so the following analyses use only
near speaker categories (i.e., CHN, FAN, MAN, CXN) as well as the
overlap category (OLN), with all other categories mapped as non-speech
(i.e., CHF, CXF, FAF, MAF, NOF, NON, OLF, TVF, SIL). For a first
analysis on all files, TVN was also mapped as non-speech; a follow-up
analysis only on ACLEW data segregated TVN such that there were 5
\enquote{speaker} categories: CHN, FAN, MAN, CXN, and TVN.

LENA's false alarm (i.e., saying that someone was speaking when they
were not) averaged 15\%, whereas the miss rate averaged 55\%. Imagining
for a moment this is a two class solution (speech versus non-speech),
then this means that the LENA system avoids \enquote{fantasizing} speech
that is not there, at the risk of missing speech that is there. This
kind of setting is preferable when prioritizing precision over recall.
We'll return to that below, when actually discussing precision and
recall of the different tags. The confusion rate, as mentioned above, is
only calculated for the correctly detected speech (i.e., not the speech
that was missed, which counts towards the miss rate, nor the speech that
was falsely identified, which is considered in the false alarm). The
confusion rate was very low, averaging 8\%. These three metrics can be
added together into a single \enquote{diarization error rate}; of
course, if one of them is NA, then DER is NA; 36\% of the clips had NA
diarization error rate (mostly due to false alarm rate being NA). The
mean diarization error rate over all other clips was 79\%. In a
secondary analysis only on the ACLEW data, \ldots{} COMPLETE\ldots{} not
sure the evaluation would be fair to LENA. My understanding is that
their human annotators marked all sound as TV -- whereas you only marked
speech as electronic. This means that neither the recall nor the
precision can be trusted in our analysis: If we find that 50\% of what
LENA called TV was tagged as electronic speech, this may well be true.
the other 50\% was music, jingles, other TV sound. If we say that the
recall is 30\%, we don't know what the LENA-defined recall was --
perhaps LENA did miss 70\% of what you tagged as electronic speech, but
found 100\% of the music and the other TV sounds, so the recall might be
much higher than what we say.

\subsection{LENA classification accuracy: Precision and
recall}\label{lena-classification-accuracy-precision-and-recall}

\begin{verbatim}
By now, we have established that the best performance (when "far" labels such as CHF and OLF are mapped onto silence), the overall relative diarization error rate is about 79%, due mainly to missing speech (55%), with false alarms (15%) and confusion between talker categories (8%) constituting a relatively small proportion of errors. However, this metric may not capture what our readers are interested in, for two reasons. First, this metric gives more importance to correctly classifying segments as speech versus non-speech (False alarms + misses) than confusing talkers (confusion). Second, many LENA adopters use the system not to make decisions on the sections labeled as non-speech, but rather on sections labeled as speech, and particularly those labeled adults and key child. The metrics above do not give more importance to these two categories, and do not give us insight on the patterns of error made by the system. Looking at precision of speech categories is crucial for users who interpret LENA's estimated quantity of adult speech or key child speech, as low precision means that some of what LENA called e.g. key child was not in fact the key child, and thus it is providing overestimates. Looking at recall may be most interesting for adopters who intend to employ LENA as a first-pass annotation: the lower the recall, the more is missed by the system and thus cannot be retrieved (because the system labeled it as something else, which will not be inspected given the original filter). Recall also impacts quantity estimates, since it indicates how much was missed of that category.
\end{verbatim}

Therefore, this subsection shows confusion matrices, containing
information on precision and recall, for each key category. For this
analysis, we collapsed over all human annotations that contained overlap
between two speakers into a category called \enquote{overlap}. Please
remember that this category is not defined the same way as the LENA
overlap category. For LENA, overlap between any two categories falls
within overlap -- i.e., CHN+TV would be counted towards overlap; whereas
for us, only overlap between two talker categories (e.g., key child and
female adult) counts as overlap. (Note that neither case contemplates
overlap between two speakers of the same category as overlap.)

We start by explaining how to interpret one cell in Figure (precision):
Focus on the crossing of the human category FEM and the LENA category
FAN; when LENA tags a given frame as FAN, this corresponds to a frame
tagged as being a female adult by the human 59\% of the time. This
category, as mentioned above, is the most common speaker category in the
audio, so that over 65k frames were tagged as being female adult by the
human and LENA. The remaining 41\% of frames were actually other
categories according to our human coders: 36\% were silence and 5\% were
confusion with other speaker tags. Inspection of the rest of the
confusion matrix shows that, other than silence, this is the most
precise LENA tag. Precision for CHN is second, at 40\%; thus, fewer than
half of the frames labeled as being the key child are, in fact, the key
child. The majority of the framesthe LENA incorrectly tagged as being
the key child are actually silence (or rather, lack of speech) according
to the human annotator (44\%), with the 16\% remaining errors being due
to confusion with other categories: About 9\% of them are actually a
female adult; 2\% are another child; and 5\% are regions of overlap
across speakers, according to our human coders. MAN and CXN score
similarly, 8 and 7\% respectively, meaning that less than a tenth of the
areas LENA tagged as being these speakers actually correspond to them.
As with the key child, most errors are due to LENA tagging silent frames
as these categories. However, in this case confusion with other speaker
tags is far from negligible. In fact, the most common speaker tag in the
human annotation among the regions that LENA tagged as being MAN were
actually female adult speech (34\%); and, for CXN, it was the key child.
In a nutshell, this suggests extreme caution before undertaking any
analyses that rely on the precision of MAN and CXN, since most of what
is being tagged as such is silence or other speakers. Another
observation is that the \enquote{far} tags of the speaker categories do
tend to more frequently correspond to what humans tagged as silence
(77\%) than the \enquote{near} tags (49\%), and thus it is reasonable to
exclude them from consideration. The relatively high proportion of near
LENA tags that correspond to regions that humans labeled as silence
could be partially due to the fact that the LENA system, in order to
process a daylong recording quickly, does not make judgments on small
frames independently, but rather imposes a minimum duration for all
speaker categories, padding with silence in order to achieve it. Thus,
any key child utterance that is shorter than .6 secs will contain as
much silence as needed to achieve this minimum (and more for the other
talker categories). Our system of annotation, whereby human annotators
had no access whatsoever to the LENA tags, puts us in an ideal situation
to assess the impact of this design decision, because any annotation
that starts from the LENA segmentation should bias the human annotator
to ignore such short interstitial silences to a greater extent than if
they have no access to their tags whatsoever. These analyses shed light
on the extent to which we can trust the LENA tags to contain what the
name indicates. We now move on to recall, which indicates a
complementary perspective: how much of the original annotations were
captured by LENA.

\begin{verbatim}
Again, we start with an example to facilitate the interpretation of this figure: The best performance for a talker category this time is CHN: Nearly half of the original frames humans tagged as being uttered by the key child were captured by the LENA under the CHN tag. Among the remaining regions that humans labeled as being the key child, 22% was captured by LENA's CXN category and 20% by its OLN tag, with the other 15% spread out across several categories. This result can be taken to suggest that an analysis pipeline that uses the LENA system to capture the key child's vocalizations by extracting only CHN regions will get nearly half of the key child's speech. Where additional human vetting is occuring in the pipeline, such researchers may consider additionally pulling out segments labeled as CXN, since this category actually contains a further 22% of the key child's speech. Moreover, as we saw above, over a third of these LENA tags corresponds to the key child, which means that human coders who are re-coding these regions could filter out the two thirds that do not.
Many colleagues also use the LENA as a first pass to capture female adult speech via their FAN label. Only a third of the female adult speech can be captured this way. Unlike the case of the key child, missed female speech is classified into many of the other categories, and thus there may not be an easy solution. However, if the hope is to capture as much of the female speech as possible, perhaps a solution may be to also pull out OLN regions, since these capture a further 24% of the original female adult speech and, out of the OLN tags, 17% are indeed female adults (meaning that human annotators re-coding these regions need to filter out 4 out of 5 clips, on average).
For the remaining two near speaker labels (MAN, CXN), recall was 15-18%, meaning that less than a quarter of male adult and other child speech is being captured by LENA. In fact, most of these speakers' contributions are being tagged by the LENA as OLN (34%) or silence (21-22%), although the remaining sizable proportion of misses is actually distributed across many categories. 
\end{verbatim}

Finally, as with precision, the \enquote{far} categories show worse
performance than the \enquote{near} ones. It is always the case that a
higher percentage of frames is \enquote{captured} by the near rather
than the far labels. For instance, out of all frames attributed to the
key child by the human annotator, 43\% were picked up by the LENA CHN
label and 0\% by the LENA CHF label. This result can be used to argue
why, when sampling LENA daylong files using the LENA software, users
need not take into account the \enquote{F} categories.

\subsection{Child Vocalization Counts (CVC)
accuracy}\label{child-vocalization-counts-cvc-accuracy}

Given the inaccuracy of far LENA tags, and in order to follow the LENA
system procedure, we only counted vocalizations attributed to CHN and
ignored those attributed to CHF. As shown in Figure (CVC), there is a
strong association between clip-level counts estimated via the LENA
system and those found in the human annotations: the Pearson correlation
between the two was .7 when all clips were taken into account, and .77
when only clips with some child speech (i.e., excluding clips with 0
counts in both LENA and human annotations) were considered. This
suggests that the LENA system captures well differences in terms of
number of child vocalizations across clips.

However, users need more: They also interpret the absolute number of
vocalizations found by LENA. Therefore, it is important to also bear in
mind the relative error rate: Given a LENA estimate, how close may the
actual number be? One issue is, as discussed above for the speech
technology metrics, relative error rates require the number in the
denominator to be non-null. For this analysis, thus, we removed the 306
clips in which the human annotator said there were no child
vocalizations whatsoever. When we do this, the mean relative error rate
is -37\% (median -52\%), indicating that the LENA underestimates the
number of vocalizations by about a third. However, the range was
considerable, going from -100\% to 700\%. A reanalysis of absolute error
rates shows quite a different pattern: FILL THIS IN

\subsection{Conversational Turn Counts (CTC)
accuracy}\label{conversational-turn-counts-ctc-accuracy}

\begin{verbatim}
Again, we only considered "near" speaker categories in the turn count, and applied the same rule the LENA does, where a turn can be from the key child to an adult or vice versa, and should happen within 5 seconds to be counted. The association between clip-level LENA and human CTC was weaker than that found for CVC: Pearson r over all clips was r=.5; and excluding clips where both the human and the LENA reported no turns led to an even lower estimate, at r=.42. Mean relative error rates excluding the 325 clips where the human annotation contained no child-adult or adult-child turns (because RER is undefined in such clips) were also more considerable for CTC than CVC (mean -89%, median -95%), although the range was narrower (-100 to 5%).
\end{verbatim}

A reanalysis of absolute error rates shows quite a different pattern:
FILL THIS IN

\subsection{Adult Word Counts
accuracy}\label{adult-word-counts-accuracy}

One child in the SOD corpus was learning French. We have included this
child to increase power, but results without this one child are nearly
identical. The association between clip-level LENA and human AWC was
strong: Pearson r over all clips was r=.75. Excluding clips where both
the human annotators and the LENA reported word counts of zero led to a
slightly lower estimate, at r=.69. Mean relative error rates excluding
the 361 clips where the human annotators said there were no words
(because RER is undefined in such clips) averaged 55\% (median -18\%),
with a considerable range (-100 to 7400\%).

\subsection{Effects of age and differences across
corpora}\label{effects-of-age-and-differences-across-corpora}

The preceding sections include results that are wholesale, over all
corpora. However, we have reason to believe that performance could be
higher for the corpora collected in North America (BER, WAR, SOD) than
those collected in other English-speaking countries (ROW) or non-English
speaking populations (TSI). Additionally, our age ranges are wide, and
in the case of TSI children, some of the children are older than the
oldest children in the LENA training set. To assess whether accuracy
varies as a function of corpora and child age, we fit mixed models as
follows. We predicted false alarm, miss, and confusion rates from
corpus, child age, and the interaction as fixed, child ID as random, on
clips where there was some speech according to the human annotator;
since misses and confusions are undefined when the annotator said there
was no speech but false alarms is not, we additionally fitted a model to
all clips (i.e., regardless of whether they had speech or not. We
followed up with an Analysis of Variance (type 3) to assess
significance. In none of these analyses was corpus, child age, or their
interaction significant. For CVC, we fit a mixed model where CVC
according to the human was predicted from CVC according to LENA, in
interaction with corpus and age, as fixed factors; with child ID as
random effect. An Analysis of Variance (type 3) found a triple
interaction, suggesting that the predicted value of LENA with respect to
human CVC depended on both the corpus and the child age; and a two-way
interaction between CVC by LENA and corpus. To investigate these
further, we fit the same regressions within each corpus separately. This
revealed that there accuracy of LENA CVC increased with age for BER, but
decreased for WAR, being stable in the others.

For CTC, we fit a mixed model where CTC according to the human was
predicted from CTC according to LENA, in interaction with corpus and
age, as fixed factors; with child ID as random effect. An Analysis of
Variance (type 3) found a two-way interaction between CTC by LENA and
corpus. To investigate this further, we fit the same regressions within
each corpus separately. These follow-up analyses revealed that CTC by
LENA was a better predictor of human-tagged CTC for WAR (t=5.06) than
ROW (t=2.04) or SOD (t=2.92), and for these than for BER (t=1.07).

\section{Discussion}\label{discussion}

\section{Acknowledgments}\label{acknowledgments}

\newpage

\section{References}\label{references}

\setlength{\parindent}{-0.5in} \setlength{\leftskip}{0.5in}






\end{document}
